\chapter*{Conclusion}
\addcontentsline{toc}{chapter}{Conclusion}

The Internet of Things presents many advantages that can bring a lot of positive things. Indeed, the IoT offers a lot of automation and thus avoid human intervention in specific situations, as well as optimizing energy consumption. However, the IoT presents a few challenges that explain why it has yet to be used more regularly. One of those challenges is the restrictions imposed by using constrained, less powerful devices than our everyday computers. This imposes new lightweight protocols. Another one is the lack of standardization in some aspects of the IoT fields. Users developed their own data format which causes huge heterogeneity.\\

The subject of our thesis was to develop a monitoring tool of Internet of Things networks. The solution we come up to was using TinyIPFIX on IoT devices to collect data about the current state of a network under analysis.\\

We have divided our thesis in three main parts.\\

In the first part, we have introduced \textit{the state of the art} of the Internet of Things. We have explained what it is, and what the smart objects composing such networks are. We have also discussed about its existing challenges that are the reasons why it is not widely deployed yet. We have also introduced the operating systems that are used in the IoT context, and why we have chosen Contiki as the main operating system. Part of the reason is due to the simulation software Cooja. To conclude the state of the art, we have introduced existing tools and technologies for the Internet of Things and also traditional networks. We have also specifically introduced Netflow (particularly IPFIX) and its principles, as it was among the technologies that were key to being able to monitor IoT networks and retrieve information about them. Additionally, we have introduced several existing monitoring tools that have been developed for traditional networks and IoT networks, and where the main differences with our own monitoring tool lie. One main difference is that the tools presented have set their sight on monitoring the sensors values while we wanted to focus on network traffic informations.\\

The second part describes our solution and how it responds to the problem of the thesis. It is divided into two chapters, chapters 4 and 5, one that explains the implementation of our software as the solution, the other one that showcases our software and explains how to use it.\\

More specifically, we have described in chapter 4 the whole architecture of our monitoring software. As explained, the architecture is divided in two subdivisions, one that is responsible for collecting data in a netflow fashion and sending it using TinyIPFIX messages, TinyIPFIX being a lightweight version of IPFIX (or Netflow v10). The TinyIPFIX messages are then reformatted into IPFIX messages by the gateway, and are then ready to be sent to the server. The second part of our archictecture is responsible for receiving and processing the collected information. The data is received from the gateway node of the IoT network to the server via UDP. Once the data is received, the server stores the messages in a database so that it can retrieve and process them. It also plays the role of a web server so as to present graphically the values stored in those messages. \\

We have also presented the technologies that we have chosen to use to develop our software and the reasons behind these choices. Indeed, we have chosen to use the technology Node.js that is quite convenient for developing HTTP servers, which is what we needed to develop a monitoring web-server. We also used the Express framework on top of Node.js. We found it convenient to develop a web-server as a software thanks to the ease with which one can install it along with accessing it, as it scales better with the number users having to use it at the same time. With the help of Javascript, we had an event-driven language, convenient for reacting promptly when the server IPFIX messages from the gateway node. Javascript also couples well with the JSON format that is convenient for dynamic websites such as ours. Along with using HTML and CSS for the visualization aspects of our software, we have used PostgreSQL as the database language for storing logs of data collected beforehand. Following the justification of our choices, we have described why our software is composed of four modules, the collector, the log, the Nodes Status, and the Web. Splitting the software into modules brings \textit{modularity} to the sotware, that can be further extended afterwards.\\

In chapter 5, we have written a guide on how to use the software and the different existing features, and depicted screenshots of it. As stated earlier, it is a web interface. There are two main tabs, one showing \textit{traffic volume} evolution through time, the other one showing a \textit{topology chart}. \\

The traffic tab contains several panels, all giving specific information about traffic volume generated in the network according to time, with specific filters such as IPFIX traffic, TinyIPFIX traffic, or normal traffic (regular packets sent between nodes). This tab also depicts specific statistics such as minimum and maximum bytes generated in a five-minutes interval and more generally some information about the network. \\

The topology tab showcases the current topology of the IoT network. The topology is an acyclic graph, more specifically a tree. All nodes composing the topology are clickable to obtain more specific information, such as its ID, its battery level, etc.\\

The third and final part was focused on the analyze of our solution. Mainly, the analyze was focused by the cost induced by the fact that nodes send meta information about the networks (the different flows). We clearly expressed this cost in term of energy consumption as one of the goal set for this thesis was to not excessively use the battery. To do so we used three different configurations depending on if the nodes send data for monitoring purposes by TinyIPFIX or had aggregators merging those TinyIPFIX packets. The results that came out were in favor of using aggregators so as to reduce the battery consumption. Also with 20 nodes, the estimated battery lifetime of nodes were reduced from 190 days to 174 days for exporters from a network using aggregators. A huge factor in the energy consumption was the radio usage time. Indeed, nodes sending TinyIPFIX packets spend more time transmitting and receiving data via radio which is quite logical.\\

Finally, this part also discusses about alternatives and possible extensions, as well as security. In this chapter, we discuss about Nfsen, a graphical web-based tool that provides a lot of monitoring information on IoT networks, using Nfdump. Nfsen was along with Nfdump the first technology that we started to work on to develop our monitoring software. However, as we have explained, we decided to use Node.js instead of Nfsen for conviniency and because it being more appropriate for the information that we wanted to retrieve from IoT networks and monitor. \\

We have also discussed about possible extensions to our software such as adding information about loads passing through links, handling network failures, and depicting protocol information on specific protocols such as UDP and RPL. We have also discussed about problems regarding security and what are the impacts of not taking security measures while monitoring the network which, in the current design of our software, is sensible to data tampering. Moreover, data is sent in the clear which would permit anyone to easily listen to the various communications.\\

In the end, those extensions have not been implemented because of them not being among the main purposes of our research. Most of the extensions require some change in the Netflow structure that we use to collect the data about flows, and require that motes have more capacities such as sending specific packets or being able to handle more situations.\\
