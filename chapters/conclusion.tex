\chapter*{Conclusion}
\addcontentsline{toc}{chapter}{Conclusion}

The Internet of Things presents many advantages that would bring a lot of positive things. The subject of our thesis was to develop a monitoring tool of Internet of Things networks, using netflow on IoT devices to collect data about the current state of a network under analysis.

We have divided our thesis in three main parts. \\

In the first part, we have introduced \textit{the state of the art} of the Internet of Things. We have explained what it is, and what the smart objects composing such networks are. We have also discussed about its existing challenges that are the reasons why it is not widely deployed yet. We have also introduced the operating systems that are used in the IoT context, and why we have chosen Contiki as the main operating system. To conclude the state of the art, we have introduced existing tools and technologies for the Internet of Things and also traditional networks. We have also specifically introduced Netflow and its principles, as it was among the technologies that were key to being able to monitor IoT networks and retrieve information about them. Additionally, we have introduced several existing monitoring tools that have been developed for traditional networks and IoT networks, and where the main differences with our own monitoring tool lie.\\

The second part describes our solution and how it responds to the problem of the thesis. More specifically, we have described the architecture 
