\chapter{State of the art} \label{ch:state_art}

In this section, we will go deeper into the world of the \textit{Internet of Things}, see what differs from the traditional networking domain and explain the impact of such differences. We will also give insight on what has been done in matter of monitoring IoT network.\\

\section{What is the Internet of Things?}
Challenges of iot : preface of book (page xx)\\
+ slide 17 du premier cours de mobile.

The Internet of Things can be used in many different situations. Power grids can now be better controlled and managed with smart objects, the temperature in buildings and houses can now be monitored and smart objects control radiators and air conditioners to modify ambient temperature, and thus make a more efficient consumption of energy. Shipping containers could also be equipped with devices to better help monitor and modify the climate inside, thus making sure food, clothing, and various materials are better taken care of.\\
\subsection{Smart Objects ? IoT Devices?}

Smart objects are the main components of an IoT Network. We will explain what makes them special and the purpose of their use.\\

Smart objects, also called IoT Devices, smart devices or connected devices, are physical objects, that are embedded with an electronical component, that allows them to communicate with other objects. Those objects are composed of a software, are equipped with a microprocessor, a sensor or an actuator that allows them to interact with the physical and modify/control it, and network connectivity to allow exchange and collection of data. Thanks to network connectivity, devices can sense and exchange sense data of the physical world with each other. They are also equipped with a small battery, that provides a power source to the device.\\

While it seems pretty straightforward, there does not seem to be a fine line of what exactly smart objects refer to. They may refer to the actual small device with processing and interacting capacity, or the bigger entity that is a physical object made "smart" by the addition of a sensor or actuator. In our case we will consider the latter definition of what an IoT device is.\\

(source : theinternetofthings.pdf, McKensey Quarterly)
The main purpose of smart objects is automation, to replace human intervention for specific computations and data collection. Obviously, they allow the \textit{collection} of bigger loads of data at a faster pace. Once data is collected, the next step is to make actual conclusions about data information. From that point, according to those computations, instructions are passed through the IoT network and the network can enter the process of modifying the physical world through actuators, in an automated fashion and again with no human intervention once the IoT network is set up.\\

While automation is already of great use, the Internet of Things aims at \textit{optimizing} how it affects the physical world. 
1. Process optimization
The Internet of Things is opening new frontiers for improving 
processes. Some industries, such as chemical production, are 
installing legions of sensors to bring much greater granularity to 
monitoring. These sensors feed data to computers, which in turn 
analyze them and then send signals to actuators that adjust processes—
for example, by modifying ingredient mixtures, temperatures, or 
pressures. Sensors and actuators can also be used to change the 
position of a physical object as it moves down an assembly line, 
ensuring that it arrives at machine tools in an optimum position 
(small deviations in the position of work in process can jam or even 
damage machine tools). This improved instrumentation, multiplied 
hundreds of times during an entire process, allows for major 
reductions in waste, energy costs, and human intervention.

2. Optimized resource consumption
Networked sensors and automated feedback mechanisms can change 
usage patterns for scarce resources, including energy and water, often 
by enabling more dynamic pricing. Utilities such as Enel in Italy and 
Pacific Gas and Electric (PG&E) in the United States, for example, 
are deploying “smart” meters that provide residential and industrial 
customers with visual displays showing energy usage and the real-
time costs of providing it. (The traditional residential fixed-price-
per-kilowatt-hour billing masks the fact that the cost of producing 
energy varies substantially throughout the day.) Based on time-of-use 
pricing and better information residential consumers could shut down 
air conditioners or delay running dishwashers during peak times.  
Commercial customers can shift energy-intensive processes and 
production away from high-priced periods of peak energy demand to 
low-priced off-peak hours.
Data centers, which are among the fastest-growing segments of global 
energy demand, are starting to adopt power-management techniques 
tied to information feedback. Power consumption is often half of a 
typical facility’s total lifetime cost, but most managers lack a detailed 
view of energy consumption patterns. Getting such a view isn’t easy, 
since the energy usage of servers spikes at various times, depending 
on workloads

\subsection{Challenges of the IoT}
\subsection{Wireless Sensor Networks}
Data transmission, security, resource constraint, 

\subsection{The IoT, Not Totally Deployed Yet?}

Most existing applications using the Internet of Things have been developed privately. Companies, developers, researchers have developed some specific solutions using the Internet of Things. However, those solutions are more specific, and hence optimized, for one particular application.\\

\section{Netflow - A Traffic Collector}
\textit{Netflow} is a feature that was created by Cisco Systems and introduced on Cisco Routers. Netflow allows the collection of the IP traffic in networks, and thus monitoring network traffic. Information such as Source and Destination addresses (defined as "Flows) and Traffic volume can be retrieved and further analyzed.\\

There are three components when having Netflow set up : the Flow Exporter, the Flow Collector(s), and the analysis application (reference wikipedia/article). The Flow Exporter collects packets and forms what we call flows (having various definitions according to the version of Netflow used). Flows represent a stream of packets sharing common attributes (such as source and destination addresses). The exporter records the flows into Netlow records and sends passes them onto the Flow Collector. After reception, the Flow Collector will store the flow records received from the Exporter. The data stored can then be analyzed by applications, hence having statistics of traffic exchanges in a particular network.\\

\todo{Ajouter des détails techniques ici sur Netflow}.\\

Having defined what Netflow is, the main goal here is to be able to use Netflow on an IoT device. In the Objective section, we explained that our main goal was to analyze the traffic of an IoT Network and retrieve pertinent information such as the volume exchanged in the network, and its topology. Netflow is basically what we want to achieve in terms of data collecting. However, it has not been implemented by Cisco for IoT devices. Our task is to implement Netflow for an IoT device, using predefined data formats that will allow us to collect the information we are interested in (battery level, source and destination addresses, packet volume, etc.). In the Solution section (fourth section?), we will describe in more details how we used Netflow in IoT Networks as a solution to our monitoring task.

\section{ContikiOS - Cooja - Softwares a portee de main}

We will now discuss Operating Systems in the Internet of Things. As discussed earlier, objects in IoT Networks have limited capabilities, i.e. processing capacity and storage, and limited battery. It is thus preferred to have an operating system that is not demanding and as light as possible. \\

Though not ideal, a light version of Linux could be used. However, one requirement of working with smart objects is having the ability to react to \textit{Real-Time events}. In a situation where a smart sensor is used in a car to make its airbags open when a car crash occurs, the software in the said sensor must react to the crash almost instantly. In this case, we need a maximum time reaction to an event, otherwise the purpose of the object (and thus the airbag) is not met. A few operating systems have been developed to answer to the requirements of IoT objects, they are light, have a minimal set of functionalities, plus they guarantee time-bounded reaction to events. (source : Mobile and embedded systems, prof Sadre)\\

\subsection{Cooja Simulating Software}
(besoin de mettre une sous-section ou pas?)

Among existing operating systems, we have chosen to use one that is called \textit{ContikiOS}. It is an open source operating system and it was developed in 2003. Contiki is quite light in terms of memory, processing speed, and communication bandwidth. It is preemptive.\\

\todo{WHY WE HAVE CHOSEN CONTIKI?}

\begin{itemize}
\item protothreads
\item Reduced C
\end{itemize}
wikipedia : "Contiki provides three network mechanisms: the uIP TCP/IP stack,[5] which provides IPv4 networking, the uIPv6 stack,[6] which provides IPv6 networking, and the Rime stack, which is a set of custom lightweight networking protocols designed for low-power wireless networks. The IPv6 stack was contributed by Cisco and was, when released, the smallest IPv6 stack to receive the IPv6 Ready certification.[7] The IPv6 stack also contains the Routing Protocol for Low power and Lossy Networks (RPL) routing protocol for low-power lossy IPv6 networks and the 6LoWPAN header compression and adaptation layer for IEEE 802.15.4 links.

Rime is an alternative network stack, for use when the overhead of the IPv4 or IPv6 stacks is prohibitive. The Rime stack provides a set of communication primitives for low-power wireless systems. The default primitives are single-hop unicast, single-hop broadcast, multi-hop unicast, network flooding, and address-free data collection. The primitives can be used on their own or combined to form more complex protocols and mechanisms.[8]" 

\subsection{Cooja Simulating Software}
An existing tool called \textit{Cooja} has been developed on Contiki. Cooja is a simulation software (partly an emulator too) for Internet of Things networks. It has many features, such as allowing to build networks with different types of components (Sky motes, Z1 motes). Those components are Contiki nodes, i.e. nodes working through ContikiOS. Cooja allows to upload code to virtual motes, the same way Contiki code may be uploaded on physical. Cooja can either emulate nodes (the hardware of each component is entirely emulated), or create "Cooja nodes" where Contiki code is uploaded on, compiled and then executed on a simulation host. (Cooja allows to use non-Contiki nodes as well. Pertinent?). Cooja presents itself as a very useful software for our thesis, as it allows us to simulate large networks, and is quite useful when it comes to testing, since the uploading and compilation time of Contiki codes on the nodes in the Network we are testing and analyzing is faster than on real hardware. It also avoids physical material restriction.

\section{Monitoring tools for traditional networks}

\subsection{Netflow - A Traffic Collector}
\textit{Netflow} is a feature that was created by Cisco Systems and introduced on Cisco Routers. Netflow allows the collection of the IP traffic in networks, and thus monitoring network traffic. Information such as Source and Destination addresses (defined as "Flows) and Traffic volume can be retrieved and further analyzed.\\

There are three components when having Netflow set up : the Flow Exporter, the Flow Collector(s), and the analysis application (reference wikipedia/article). The Flow Exporter collects packets and forms what we call flows (having various definitions according to the version of Netflow used). Flows represent a stream of packets sharing common attributes (such as source and destination addresses). The exporter records the flows into Netlow records and sends passes them onto the Flow Collector. After reception, the Flow Collector will store the flow records received from the Exporter. The data stored can then be analyzed by applications, hence having statistics of traffic exchanges in a particular network.\\

Ajouter des détails techniques ici sur Netflow.\\

Having defined what Netflow is, the main goal here is to be able to use Netflow on an IoT device. In the Objective section, we explained that our main goal was to analyze the traffic of an IoT Network and retrieve pertinent information such as the volume exchanged in the network, and its topology. Netflow is basically what we want to achieve in terms of data collecting. However, it has not been implemented by Cisco for IoT devices. Our task is to implement Netflow for an IoT device, using predefined data formats that will allow us to collect the information we are interested in (battery level, source and destination addresses, packet volume, etc.). In the Solution section (fourth section?), we will describe in more details how we used Netflow in IoT Networks as a solution to our monitoring task.

\section{Current Monitoring Tools for IoT}
