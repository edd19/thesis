\chapter{State of the art} \label{ch:state_art}

In this section, we will go deeper into the world of the \textit{Internet of Things}, see what differs from the traditional networking domain and explain the impact of such differences. We will also give insight on what has been done in matter of monitoring IoT network.\\

\section{What is the Internet of Things?}

The Internet of Things or IoT, also called Internet of Everything, Web of Things, can be viewed as the interconnection of things, often referred as objects, motes or smart objects. The Global Standards Initiative on Internet of Things (IoT-GSI) defined \textit{Internet of Things} as "a global infrastructure for the information society, enabling advanced services by interconnecting (physical and virtual) things based on existing and evolving interoperable information and communication technologies"\cite{ituitu}. They also give an explanation about what they consider a \textit{"thing"} to be. A thing is "an object of the physical world (physical things) or the information world (virtual things), that is capable of being identified and integrated into communication networks"\cite{ituitu}. \\

\todo[inline]{Give other examples.}
With IoT, we dispose of networks constituted by devices which have the ability to communicate between them to exchange data. Also those devices can interact with their environment with means of sensors or actuators. An illustration of such network would be the \textit{Smart Home}. With Smart Home, automation comes into our house. The idea is to have, for example, your door talking to the lights to tell them to light on or to light off depending on if you enter or leave the room. You could have that your alarm would tell the coffee machine to prepare some coffee. Also, you could remotely manage your house by activating or deactivating the heat and so on. \\

It is often viewed as the next \textit{Industrial Revolution} which is named as \textit{Industry 4.0}. This revolution is made possible by the emergence of IoT but also Cloud Computing which make possible the treatment of the vast amount of data that comes from IoT networks.\\

Challenges of iot : preface of book (page xx)\\
+ slide 17 du premier cours de mobile.

The Internet of Things can be used in many different situations. Power grids can now be better controlled and managed with smart objects, the temperature in buildings and houses can now be monitored and smart objects control radiators and air conditioners to modify ambient temperature, and thus make a more efficient consumption of energy. Shipping containers could also be equipped with devices to better help monitor and modify the climate inside, thus making sure food, clothing, and various materials are better taken care of.\\
\subsection{Smart Objects ? IoT Devices?}

Smart objects are the main components of an IoT Network. We will explain what makes them special and the purpose of their use.\\

Smart objects, also called IoT Devices, smart devices or connected devices, are physical objects, that are embedded with an electronical component, that allows them to communicate with other objects. Those objects are composed of a software, are equipped with a microprocessor, a sensor or an actuator that allows them to interact with the physical and modify/control it, and network connectivity to allow exchange and collection of data. Thanks to network connectivity, devices can sense and exchange sense data of the physical world with each other. They are also equipped with a small battery, that provides a power source to the device.\\

While it seems pretty straightforward, there does not seem to be a fine line of what exactly smart objects refer to. They may refer to the actual small device with processing and interacting capacity, or the bigger entity that is a physical object made "smart" by the addition of a sensor or actuator. In our case we will consider the latter definition of what an IoT device is.\\

(source : theinternetofthings.pdf, McKensey Quarterly)
The main purpose of smart objects is automation, to replace human intervention for specific computations and data collection. Obviously, they allow the \textit{collection} of bigger loads of data at a faster pace. Once data is collected, the next step is to make actual conclusions about data information. From that point, according to those computations, instructions are passed through the IoT network and the network can enter the process of modifying the physical world through actuators, in an automated fashion and again with no human intervention once the IoT network is set up.\\

To go along with automation, the Internet of Things aims at \textit{optimizing} how it affects the physical world, in terms of energy consumption efficiency, timing, accuracy, etc.

\subsection{Challenges of the IoT}

page 2 article : IoT : a survey (deuxième colonne) :
It shall not be forgotten, anyway, that the words ‘‘Inter-
144
net” and ‘‘Things”, when put together, assume a meaning
145
which introduces a disruptive level of innovation into to-
146
day ICT world. In fact, ‘‘Internet of Things” semantically
147
means ‘‘a world-wide network of interconnected objects
148
uniquely addressable, based on standard communication
149
protocols”
[3]. This implies a huge number of (heteroge-
150
neous) objects involved in the process.
151
The object unique addressing and the representation
152
and storing of the exchanged information become the most
153
challenging issues, bringing directly to a third, ‘‘
Semantic
154
oriented”, perspective of IoT. "
\subsection{Wireless Sensor Networks}

(en.wikipedia.org/wiki/Wireless\_sensor\_network),\\

The main kind of IoT network we have focused on are Wireless Sensor Networks. While our solution fits most types of IoT networks, we have decided to mainly focus on WSNs because they are the most constraining types(développer?).\\

(per Connecting iot objets with IP book)
The idea behind \textit{Wireless Sensor Networks} (\textit{WSN} in shortened form) is having small sensors collecting pertinent data from the physical world, communicating wirelessly to other sensors and in the end to a base station where all the data is collected and stored for further processing. Sensors not capable of directly transmitting information to the base station send their data to other sensors capable of relaying the information towards it (AJOUTER UNE IMAGE DE WSN?), in a hop by hop fashion.\\

Wireless Sensor Networks are regularly used for environmental conditions monitoring such as wild fire tracking, agriculture management, humidity and temperature monitoring in forests, effect of global warming on glacier structures, air pollution monitoring, and many more. They are also used in the industrial field such as in data centers, mainly to monitor the temperature of racks.  \\

(en.wikipedia.org/wiki/Sensor\_node),
The components of WSNs are "\textit{Wireless Nodes}" or "Motes". WSNs may be composed of a few hundreds to thousands of nodes, each node being conneted to one or several sensors. In addition to sensors, those nodes are also equipped with a microcontroller, a radio transceiver, memory, and a power source (a battery). The sensors are responsible for capturing data from the environment, especially when there is a change in an environmental condition such as temperature or humidity. (The continual analog signal produced by the sensors is digitized by an analog-to-digital converter and sent to controllers for further processing). Sensors are usually very small and thus consume a very low amount of battery power. The \textit{microcontroller} is responsible for peforming specific tasks and processing the data collected by the sensor, but also controlling other components of the node. The transceiver is used for communication purposes, mainly sending and receiving data using specific. It is often of the form of a radio or infrared. Transceivers have the particularity of both being able to send and receive data. Finally, the power source provides power supply instead of mains supply. The power source is thus in the form of a battery, it is consumed when the node is sending the environment, communicating with other sensor nodes, and doing some data processing. All of the components are designed to be as little power-consuming as possible. Having nodes not connected to mains power is an advantage because of not having to be plugged to a power supply (when the physical world makes it hard to have some, in moutains for example), but with such an advantage come resource constraints as we will explain later.  \\

The main challenges of WSNs are Data transmission, resource constraint, and security. First, nodes must send their data to the server. As discussed earlier, each node may not have the server in their sending range because of a lack of power, thus they must send their data to an intermediate node relaying the information. Moreover, limited battery, must stay idle, ENERGY HARVESTING etc. Puis security.\\

((((((Each such sensor network node has typically several parts: a radio transceiver with an internal antenna or connection to an external antenna, a microcontroller, an electronic circuit for interfacing with the sensors and an energy source, usually a battery or an embedded form of energy harvesting. A sensor node might vary in size from that of a shoebox down to the size of a grain of dust, although functioning "motes" of genuine microscopic dimensions have yet to be created. The cost of sensor nodes is similarly variable, ranging from a few to hundreds of dollars, depending on the complexity of the individual sensor nodes. Size and cost constraints on sensor nodes result in corresponding constraints on resources such as energy, memory, computational speed and communications bandwidth. The topology of the WSNs can vary from a simple star network to an advanced multi-hop wireless mesh network. The propagation technique between the hops of the network can be routing or flooding.)))))\\

\subsection{The IoT, Not Totally Deployed Yet?}

Most existing applications using the Internet of Things have been developed privately. Companies, developers, researchers have developed some specific solutions using the Internet of Things. However, those solutions are more specific, and hence optimized, for one particular application.\\

\section{Netflow - A Traffic Collector}
\textit{Netflow} is a feature that was created by Cisco Systems and introduced on Cisco Routers. Netflow allows the collection of the IP traffic in networks, and thus monitoring network traffic. Information such as Source and Destination addresses (defined as "Flows) and Traffic volume can be retrieved and further analyzed.\\

There are three components when having Netflow set up : the Flow Exporter, the Flow Collector(s), and the analysis application (reference wikipedia/article). The Flow Exporter collects packets and forms what we call flows (having various definitions according to the version of Netflow used). Flows represent a stream of packets sharing common attributes (such as source and destination addresses). The exporter records the flows into Netlow records and sends passes them onto the Flow Collector. After reception, the Flow Collector will store the flow records received from the Exporter. The data stored can then be analyzed by applications, hence having statistics of traffic exchanges in a particular network.\\

\todo{Ajouter des détails techniques ici sur Netflow}.\\

Having defined what Netflow is, the main goal here is to be able to use Netflow on an IoT device. In the Objective section, we explained that our main goal was to analyze the traffic of an IoT Network and retrieve pertinent information such as the volume exchanged in the network, and its topology. Netflow is basically what we want to achieve in terms of data collecting. However, it has not been implemented by Cisco for IoT devices. Our task is to implement Netflow for an IoT device, using predefined data formats that will allow us to collect the information we are interested in (battery level, source and destination addresses, packet volume, etc.). In the Solution section (fourth section?), we will describe in more details how we used Netflow in IoT Networks as a solution to our monitoring task.

\section{ContikiOS - Cooja - Softwares a portee de main}

We will now discuss Operating Systems in the Internet of Things. As discussed earlier, objects in IoT Networks have limited capabilities, i.e. processing capacity and storage, and limited battery. It is thus preferred to have an operating system that is not demanding and as light as possible. \\

Though not ideal, a light version of Linux could be used. However, one requirement of working with smart objects is having the ability to react to \textit{Real-Time events}. In a situation where a smart sensor is used in a car to make its airbags open when a car crash occurs, the software in the said sensor must react to the crash almost instantly. In this case, we need a maximum time reaction to an event, otherwise the purpose of the object (and thus the airbag) is not met. A few operating systems have been developed to answer to the requirements of IoT objects, they are light, have a minimal set of functionalities, plus they guarantee time-bounded reaction to events. (source : Mobile and embedded systems, prof Sadre)\\

\subsection{Cooja Simulating Software}
(besoin de mettre une sous-section ou pas?)

Among existing operating systems, we have chosen to use one that is called \textit{ContikiOS}. It is an open source operating system and it was developed in 2003. Contiki is quite light in terms of memory, processing speed, and communication bandwidth. It is preemptive.\\

\todo{WHY WE HAVE CHOSEN CONTIKI?}

\begin{itemize}
\item protothreads
\item Reduced C
\end{itemize}
wikipedia : "Contiki provides three network mechanisms: the uIP TCP/IP stack,[5] which provides IPv4 networking, the uIPv6 stack,[6] which provides IPv6 networking, and the Rime stack, which is a set of custom lightweight networking protocols designed for low-power wireless networks. The IPv6 stack was contributed by Cisco and was, when released, the smallest IPv6 stack to receive the IPv6 Ready certification.[7] The IPv6 stack also contains the Routing Protocol for Low power and Lossy Networks (RPL) routing protocol for low-power lossy IPv6 networks and the 6LoWPAN header compression and adaptation layer for IEEE 802.15.4 links.

Rime is an alternative network stack, for use when the overhead of the IPv4 or IPv6 stacks is prohibitive. The Rime stack provides a set of communication primitives for low-power wireless systems. The default primitives are single-hop unicast, single-hop broadcast, multi-hop unicast, network flooding, and address-free data collection. The primitives can be used on their own or combined to form more complex protocols and mechanisms.[8]"

\subsection{Cooja Simulating Software}
An existing tool called \textit{Cooja} has been developed on Contiki. Cooja is a simulation software (partly an emulator too) for Internet of Things networks. It has many features, such as allowing to build networks with different types of components (Sky motes, Z1 motes). Those components are Contiki nodes, i.e. nodes working through ContikiOS. Cooja allows to upload code to virtual motes, the same way Contiki code may be uploaded on physical. Cooja can either emulate nodes (the hardware of each component is entirely emulated), or create "Cooja nodes" where Contiki code is uploaded on, compiled and then executed on a simulation host. (Cooja allows to use non-Contiki nodes as well. Pertinent?). Cooja presents itself as a very useful software for our thesis, as it allows us to simulate large networks, and is quite useful when it comes to testing, since the uploading and compilation time of Contiki codes on the nodes in the Network we are testing and analyzing is faster than on real hardware. It also avoids physical material restriction.

\section{Monitoring tools for traditional networks}

\subsection{Netflow - A Traffic Collector}
\textit{Netflow} is a feature that was created by Cisco Systems and introduced on Cisco Routers. Netflow allows the collection of the IP traffic in networks, and thus monitoring network traffic. Information such as Source and Destination addresses (defined as "Flows) and Traffic volume can be retrieved and further analyzed.\\

There are three components when having Netflow set up : the Flow Exporter, the Flow Collector(s), and the analysis application (reference wikipedia/article). The Flow Exporter collects packets and forms what we call flows (having various definitions according to the version of Netflow used). Flows represent a stream of packets sharing common attributes (such as source and destination addresses). The exporter records the flows into Netlow records and sends passes them onto the Flow Collector. After reception, the Flow Collector will store the flow records received from the Exporter. The data stored can then be analyzed by applications, hence having statistics of traffic exchanges in a particular network.\\

Ajouter des détails techniques ici sur Netflow.\\

Having defined what Netflow is, the main goal here is to be able to use Netflow on an IoT device. In the Objective section, we explained that our main goal was to analyze the traffic of an IoT Network and retrieve pertinent information such as the volume exchanged in the network, and its topology. Netflow is basically what we want to achieve in terms of data collecting. However, it has not been implemented by Cisco for IoT devices. Our task is to implement Netflow for an IoT device, using predefined data formats that will allow us to collect the information we are interested in (battery level, source and destination addresses, packet volume, etc.). In the Solution section (fourth section?), we will describe in more details how we used Netflow in IoT Networks as a solution to our monitoring task.

\section{Current Monitoring Tools for IoT}
