\part{Analysis}  \label{part:analysis}

\chapter{Testbed}

In this chapter, we will present the methodology used to analyze our solution. Our tests were focused on the data retrieval as we considered it to be the part that needed to be evaluated the most. This is mainly due to the fact that our solution will be considered viable only if it do not stress intensively the network. Because we are talking about Internet of Things, and particularly Wireless Sensors Network, preserving the battery life of nodes is a must. This can be done by maximizing the CPU sleep time, or reducing the transmission or reception time. For those particular reasons, four criteria designed our tests:

\begin{itemize}
  \item CPU sleep time or its inverse the CPU duty time
  \item Radio transmission time (Tx)
  \item Radio reception time(Rx)
  \item Amount of packets induced by our solution \\
\end{itemize}

\todo[inline]{Add formula for energy consumption}
The three first arguments can give an estimate of the energy consumption. With that we can build a model to tell whether if our solution depletes the battery too much.\
To implement our benchmarks, we used the couple Contiki-OS and Cooja. The former was used to configure the nodes while the latter was used to simulate those nodes. For more details on those two, see Chap.\ref{chap:contiki}.\\

This chapter will be organized in two sections. First, we will explain the different configurations (number and role of nodes) used for our benchmarks. Secondly, we will explain in more detail the mechanisms used for nodes to send their data. The code used to implement those parts can be found on our public repository \textit{\href{https://github.com/edd19/netflow_contiki}{netflow\_contiki}}.

\section{Configurations}

Four different configurations were used for the purpose of our tests. They all differ by the facts that nodes send IPFIX messages or TinyIPFIX messages  or simply none. However they all share the common components of WSN: sensor nodes and gateway node.

\begin{description}
  \item[Simple] In this configuration, the nodes do not send information about the flows they observed or any meta information about their network. They only occupy the role of sensor nodes, meaning sensing the environment and transmitting the values observed.
  \item[Ipfix] This configuration is an upgrade of the previous one where the nodes now send informations about the flows they observerd. Those informations are formatted using the full IPFIX format. The gateway node do not need to do any conversion and the data is hence transmitted directly to the collector.
  \item[TinyIpfix] This configuration is also an upgrade of the \textit{Simple} configuration. The nodes send flows informations but using the TinyIPFIX format. This imply that the gateway node do conversion of TinyIPFIX to IPFIX before transmitting the data to the collector.
  \item[Aggregation] This last configuration adds aggregators to the \textit{TinyIpfix} configuration. Aggreagators collect TinyIPFIX messages and merge them in one message. They then send those merged messages to the gateway who will convert them to compliant IPFIX messages. In return the gateway node will send the converted messages directly to the collector. \\
\end{description}

We tested each configuration with an increasing number of nodes of 5, 10, 15 and 20. Each configuration was tested for 25 minutes. The number of nodes was limited by the computer used during benchmarking as it proved unable to simulate more than 20 nodes with Cooja. We also tested the influence of the sending time of the IPFIX messages with a 1 and 5 minutes interval between each message.\\

Also for each simulation we keep track of the following informations:
\begin{itemize}
  \item The 4-uple (Rx, Tx, Cpu, Lpm) where Rx is the radio reception time, Tx the radio transmission time, Cpu the time the Cpu is working and Lpm the time the Cpu is sleeping. This 4-uple was recorded for each minute. This gave an idea on how the energy consumption evolves minute by minute.
  \item A log of all the packets sent during the simulation thanks to a Cooja plug-in. This permits to have an idea on the amount of traffic generated by our solution.\\
\end{itemize}

Our tests were done solely on simulation using the Cooja simulation software. The nodes used during the simulations were the Zolertia Z1.

\section{Sending of data}

This section will answer the questions relative on how the motes send IPFIX/TinyIPFIX messages, what triggers the sending of those meta informations and what particular informations are sent.


\chapter{Results}

After having presented our methodology used when testing our solution, this chapter will present and discuss the results obtained from it.
