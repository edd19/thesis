\part{Analysis}  \label{part:analysis}

\chapter{Testbed}

In this chapter, we will present the methodology used to analyze our solution. Our tests were focused on the data retrieval as we considered it to be the part that needed to be evaluated the most. This is mainly due to the fact that our solution will be considered viable only if it do not stress intensively the network. Because we are talking about Internet of Things, and particularly Wireless Sensors Network, preserving the battery life of nodes is a must. This can be done by maximizing the CPU sleep time, or reducing the transmission or reception time. For those particular reasons, five criteria designed our tests:

\begin{itemize}
  \item CPU duty time
  \item CPU low power mode time
  \item Radio transmission time (Tx)
  \item Radio reception time(Rx)
  \item Amount of packets induced by our solution \\
\end{itemize}

The four first arguments can give an estimate of the energy consumption. With that we can build a model to tell whether if our solution depletes the battery too much. Dunkel et al. \cite{dunkels2007software} demonstrate that energy consumption can be given by the following formula

\begin{equation}
  E = (I_m t_m + I_l t_l + I_t t_t + I_r t_r) * V
\end{equation}

where $I_m$ and $t_m$ represents the current draw of the microprocessor when running and the time in which it has been running respectively. $I_l$ and $t_l$  are the current draw and the time of the microprocessor in low power mode. $I_t$ and $t_t$ are the current draw and the time of the communication device in transmit mode while $I_r$ and $t_r$ are the current draw and the time of the communication device in receive mode. Finally, $V$ is the supply voltage. The $I_m$, $I_l$, $I_t$, $I_r$ and $V$ values are device dependent. Different models can present different values. The Tab.\ref{table:device_consumption} shows such values for different models.\\

\begin{table}
  \centering

  \begin{tabular}{|c|c|c|c|c|c|}
    \hline
    Model & $I_m$ (mA) & $I_l$ (mA) & $I_t$ (mA) & $I_r$ (mA) & $V$ (V)\\
    \hline
    Zolertia Z1 & 5 & 0.0005 & 17.4 & 18.8 & 3 \\
    \hline
    Tmote Sky & 1.8 & 0.0545 & 19.5 & 21.8 & 3 \\
    \hline
  \end{tabular}
  \caption{Devices electrical characteristics}
  \label{table:device_consumption}
\end{table}

To implement our benchmarks, we used the couple Contiki-OS and Cooja. The former was used to configure the nodes while the latter was used to simulate those nodes. For more details on those two, see Chap.\ref{chap:contiki}.\\

This chapter will be organized in two sections. First, we will explain the different configurations (number and role of nodes) used for our benchmarks. Secondly, we will explain in more detail the mechanisms used for nodes to send their data. The code used to implement those parts can be found on our public repository \textit{\href{https://github.com/edd19/netflow_contiki}{netflow\_contiki}}.

\section{Configurations}

Four different configurations were used for the purpose of our tests. They all differ by the facts that nodes send IPFIX messages or TinyIPFIX messages  or simply none. However they all share the common components of WSN: sensor nodes and gateway node.

\begin{description}
  \item[Simple] In this configuration, the nodes do not send information about the flows they observed or any meta information about their network. They only occupy the role of sensor nodes, meaning sensing the environment and transmitting the values observed.
  \item[Ipfix] This configuration is an upgrade of the previous one where the nodes now send informations about the flows they observerd. Those informations are formatted using the full IPFIX format. The gateway node do not need to do any conversion and the data is hence transmitted directly to the collector.
  \item[TinyIpfix] This configuration is also an upgrade of the \textit{Simple} configuration. The nodes send flows informations but using the TinyIPFIX format. This imply that the gateway node do conversion of TinyIPFIX to IPFIX before transmitting the data to the collector.
  \item[Aggregation] This last configuration adds aggregators to the \textit{TinyIpfix} configuration. Aggreagators collect TinyIPFIX messages and merge them in one message. They then send those merged messages to the gateway who will convert them to compliant IPFIX messages. In return the gateway node will send the converted messages directly to the collector. \\
\end{description}

The Tab.\ref{table:configurations} resumes the different configurations used during our benchmarks. \\

\begin{table}
  \centering
  \begin{tabular}{|c|c|c|}
    \hline
    Configuration & Monitoring deployed & Aggregator \\
    \hline
    Simple & None & No \\
    \hline
    Ipfix & IPFIX & No \\
    \hline
    TinyIpfix & TinyIPFIX & No \\
    \hline
    Aggregation & TinyIPFIX & Yes \\
    \hline
  \end{tabular}
  \caption{Configurations used}
  \label{table:configurations}
\end{table}

We tested each configuration with an increasing number of nodes of 5, 10, 15 and 20. Each configuration was tested for 25 minutes. The number of nodes was limited by the computer used during benchmarking as it proved unable to simulate more than 20 nodes with Cooja. We also tested the influence of the sending time of the IPFIX messages with a 1 and 5 minutes interval between each message.\\

Also for each simulation we keep track of the following informations:
\begin{itemize}
  \item The 4-uple (Rx, Tx, Cpu, Lpm) where Rx is the radio reception time, Tx the radio transmission time, Cpu the time the Cpu is working and Lpm the time the Cpu is sleeping. This 4-uple was recorded for each minute. This gave an idea on how the energy consumption evolves minute by minute.
  \item A log of all the packets sent during the simulation thanks to a Cooja plug-in. This permits to have an idea on the amount of traffic generated by our solution.\\
\end{itemize}

Our tests were done solely on simulation using the Cooja simulation software. The nodes used during the simulations were the Zolertia Z1.

\section{Sending of data}

This section will answer the questions relative on how the motes send IPFIX/TinyIPFIX messages, what triggers the sending of those meta informations and what particular informations are sent. Distinction must be done on the three types of motes: exporter, aggregator and gateway.\\

\begin{description}
  \item[Exporter] It is straighforward. It first send the templates then set a timer. When the timer expired or when the flows table reached its maximum capacity, the exporter exports the records. All packets, templates or data, are sent whether directly to the gateway or to an aggregator node depending on the configuration. The Alg.\ref{algo:exporter} is the algorithm corresponding to an exporter.

  \begin{algorithm}
    \textbf{Function} Exporter:\\
    send templates\;
    set timer\;
    \While{true}{
      yield process until event\;
      \If{timer timeout or flows table is full}{
        send data/records\;
        empty flows table\;
        reset timer\;
      }
    }
   \caption{Exporter algorithm}
   \label{algo:exporter}
  \end{algorithm}

  \item[Aggregator] It is an extension of the Exporter. It both merge and send TinyIPFIX messages. At first it send the templates used then also set a timer. When the aggregator received an TinyIPFIX message, it merged that one with previous ones if any. When a timeout of the timer occured or when the flows table of the aggregator reached its limit, it will merge its records with the ones received then will send the new formed message to the gateway. This behavior is shown by the Alg.\ref{algo:aggregator}.

  \begin{algorithm}
    \textbf{Function} Aggregator:\\
    send templates\;
    set timer\;
    \While{true}{
      yield process until event\;
      \If{received TinyIPFIX message}{
        merge message with previous ones\;
      }
      \If{timer timeout or flows table is full}{
        merge aggregator data/records with received ones\;
        send merged messages\;
        empty flows table\;
        reset timer\;
      }
    }
   \caption{Aggregator algorithm}
   \label{algo:aggregator}
  \end{algorithm}

  \item[Gateway] It convert TinyIPFIX messages received into IPFIX message. The converted messages are sent directly to the collector. This is shown by Alg.\ref{algo:gateway}.

  \begin{algorithm}
    \textbf{Function} Gateway:\\
    \While{true}{
      yield process until event\;
      \If{received TinyIPFIX message}{
        convert into IPFIX message;
        send converted message;
      }
    }
   \caption{Gateway algorithm}
   \label{algo:gateway}
  \end{algorithm}

\end{description}

Two templates were used during our tests. The first template (Tab.\ref{table:traffic_template}) was used to send traffic information about the network while the second one (Tab.\ref{table:meta_info_template}) to obtain meta-information about nodes. The first template considers records of size of 8 bytes. Each record contains the source node id, the destination node id which is the receiving end of a flow and also the number of octets for that particular flows plus the number of packets. This template is the most often used. The second template with records of size of 5 bytes is comprised of the source node id but also the parent node id (in RPL routing) and also the remaining battery of the node. This template is not sent often. It must be sent on update of the parent or at interval of one hour. The fields source node id, destination node id, parent node id and battery were defined for the purpose of this thesis. They are not standard ones.

\begin{table}
  \centering
  \begin{tabular}{|c|c|}
    \hline
    Source Node Id & 2 bytes \\
    \hline
    Destination Node Id & 2 bytes \\
    \hline
    Octets delta count & 2 bytes \\
    \hline
    Packets delta count & 2 bytes \\
    \hline
  \end{tabular}
  \caption{Traffic template}
  \label{table:traffic_template}
\end{table}

\begin{table}
  \centering
  \begin{tabular}{|c|c|}
    \hline
    Source Node Id & 2 bytes \\
    \hline
    Parent node id & 2 bytes \\
    \hline
    Battery & 1 byte \\
    \hline
  \end{tabular}
  \caption{Meta-information template}
  \label{table:meta_info_template}
\end{table}

\chapter{Results}

After having presented our methodology used when testing our solution, this chapter will present and discuss the results obtained from it.
