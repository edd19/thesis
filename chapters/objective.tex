\chapter{Objective} \label{ch:objective}

In this section, we will explain with more details the main subject of our master thesis. The subject is defined as follows : \textit{create a tool for Network Administration to visualize an \textbf{IoT} Network}.\\

In this regard, we must create a software that is able to analyse the traffic that flows through an IoT Network. The traffic we are interested in is composed of all the data, packets, messages exchanged between the objects or that comes out of the IoT network or came into it. By visualize, we mean showing with the help of graphical content, such as curves and topologies, various information types that have been retrieved from the network. The information that we want to retrieve and process is traffic volume, battery consumption and level, and the general routing topology of the IoT network we are administrating. \\

One may ask what motivates such a research and development. IoT becomes more prevalent as of today but no tools exists to keep track of all the traffic that emerged from it. Some softwares do exist to monitor IoT, but they focus mainly on data (sensors values) that comes out of it and overlook the internal traffic generated by such networks. But the more complex IoT networks will become, the more organizations will be in need of means to observe the network entirely as to, for example, improve its efficiency. Monitoring techniques already exist for traditional networks such as \textit{Netflow} and \textit{sFlow}. However those two approaches are in their current design not directly applicable to IoT networks. They need a lot of resources that smart objects, motes do not dispose of. Thus leading to the subject of our thesis to design a tool to permit network administrators to monitor their IoT networks.\\

As we will explain later on in more detail in Chapter \ref{ch:state_art}, the Internet of Things is the interconnection of smart objects. However, those smart objects differ in a few ways from the components of "traditional" networks, that are mainly composed of powerful computers and routers.\\

Those traditional networks usually have infinite power supply since they are directly plugged to power plugs. Smart objects may be embedded to another machine, in that case they may be supplied by power. However, some objects whose purpose is to sense humidity or temperature, for example in forests attached on trees, have limited battery life.  Such network are called \textit{Wireless sensor networks} (WSN) which are capable in an automatic fashion to report periodically the environmental conditions. Therefore, the objects cannot be in a running state perpetually. Occasionally, they will be left in a standby mode, when no information is gathered, meaning the object will not use its sensor to sense the physical world, or use its actuator to control it. That way the battery life is extended, and the object is in a running period only when data must be collected or must communicate with its neighbors. \\

Another important difference is processing capacity. Components of traditional networks are usually equipped with powerful processors and thus have a stronger processing capacity. On the other hand, smart objects first have a limited size, usually a few cubic centimeters. Consequently, those smart objects have small microprocessors and also have limited storage capacity. Having these limited capabilities greatly affects how the protocols are built, as an object receiving too many packets and having too much data to process will rapidly be overworked because of its relatively weak processing capacity. \\

As a result of all those main differences, IoT Networks impose different constraints than traditional networks, and lightweight protocols must be built, developed according to those constraints.\\

The software we wanted to develop was a Web page that would display the current topology of our Network we are administrating, with various information on the traffic flowing through the network, with information about senders and receivers of the packets, along with the volume of data exchanged through a certain period of time. Additional informations as current battery of motes or their topological links would be also relevant for our case. For that, we used \textit{Contiki} with \textit{Cooja} to program motes and simulate networks. We then used \textit{Nfdump}/\textit{Nfsen} as to provide a plug-in for visualizing the informations that we sent. Additional details on these technologies will be explained on the Chapter \ref{ch:state_art}.
