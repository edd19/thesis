\chapter{Objective}

\begin{itemize}
	\item Monitoring of IoT internal traffic
	\item Talk about constraints (inherent to IoT)
	\item Reason why
\end{itemize}

TODO : parler des Wireless sensor networks, parler des Z1/sky motes ? \\

In this section, we will explain with more details the main subject of our master thesis. The subject is defined as follows : \textit{create a tool for Network Administration to visualize an \textbf{IoT} Network}.\\

In this regard, we must create a software that is able to analyse the traffic that flows through an IoT Network. By visualize, we mean showing with the help of graphical content, such as curves and topologies, various information types that have been retrieved from the network.\\

The information that we want to retrieve and process is traffic volume, battery consumption and level, and the general topology of the IoT network we are administrating.\\

One may ask what motivates such a research and development. As we will explain later on in more detail, the Internet of Things is the interconnection of smart objects. However, those smart objects differ in a few ways from the components of "traditional" networks, that are mainly composed of powerful computers and routers.\\

Those traditional networks usually have infinite power supply since they are directly plugged to power plugs. Smart objects may be embedded to another machine, in that case they may be supplied by power. However, some objects whose purpose is to sense humidity or temperature, for example in forests attached on trees, have limited battery life. Therefor, the objects cannot be in a running state perpetually. Occasionally, they will be left in a standby mode, when no information is gathered, meaning the object will not use its sensor to sense the physical world, or use its actuator to control it. That way the battery life is extended, and the object is in a running period only when data must be collected. \\

Another important difference is processing capacity. Components of traditional networks are usually equipped with powerful processors and thus have a stronger processing capacity. On the other hand, smart objects first have a limited size, usually a few cubic centimeters. Consequently, those smart objects have small microprocessors and also have limited storage capacity. Having these limited capabilities greatly affects how the protocols are built, as an object receiving too many packets and having too much data to process will rapidly be overworked because of its relatively weak processing capacity. \\

As a result of all those main differences, IoT Networks impose different constraints than traditional networks, and protocols must be built(developed?) according to those constraints.\\

Monitoring tools already exist for \\

(Such tool already exists for traditional networks, such as "Netflow tool" thing.)\\

The final software we aimed to develop was a Web page that would display the current topology of our Network we are administrating, with various information on the traffic flowing through the network, with information about senders and receivers of the packets, along with the volume of data exchanged through a certain period of time.
