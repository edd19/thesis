\chapter*{Introduction}
\addcontentsline{toc}{chapter}{Introduction}

Computers, the Internet. Those technologies are now everyday tools that a vast majority of people own and use these days, every day. They have shaped our habits, being able to communicate with everybody and anybody, through technologies other than traditional phones, with Facebook or WhatsApp for instance. The internet is now available on smaller objects, more famously on smartphones, and smart tablets. But what if other electronic devices could be connected to the Internet, and could communicate and exchange information? (What if instead of having workers collect data in a particular location, electronic devices could do this task for us?)\\

Such technologies go by the name of \textit{The Internet of Things}. The Internet of Things or IoT is defined as the interconnection of smart devices, or \textit{objects}. Typically, the IoT is composed of smart devices, also called motes which are able to communicate between themselves to exchange data for example. They can also be able to interact with the environment by sensing it with sensors (observing the temperature of a house, for example) or modify it by means of actuators (turn the heat on of a house for example).\\

As our journey as students is coming to an end, the Master thesis presents the most important step towards completion of our Master Degree.  Having chosen the Computer Network option in our Master Degree program, we were eager to make some deeper researches on networks and especially the IoT, fueled by the fact that it is a field that is not completely developed to its full potential. Hence, we have chosen the subject \textit{Network Administration Tool For the Internet of Things}, a subject suggested by Professor Sadre, with whom we have decided to work with on the thesis. \\

This thesis will explore the Internet of Things, specifically the monitoring of internetworks of such \textit{smart objects}.  In this regard, we must create a software that is able to analyse the traffic that flows through an IoT Network. The traffic we are interested in is composed of all the data, packets, messages exchanged between the objects or that comes out of the IoT network or came into it. By visualize, we mean showing with the help of graphical content, such as curves and topologies, various information types that have been retrieved from the network. The information that we want to retrieve and process is traffic volume, battery consumption and level, and the general routing topology of the IoT network we are administrating. \\

One may ask what motivates such a research and development. IoT becomes more prevalent as of today but no tools exists to keep track of all the traffic that emerged from it. Some softwares do exist to monitor IoT, but they focus mainly on data (sensors values) that comes out of it and overlook the internal traffic generated by such networks. But the more complex IoT networks will become, the more organizations will be in need of means to observe the network entirely as to, for example, improve its efficiency. Monitoring techniques already exist for traditional networks such as \textit{Netflow} and \textit{sFlow}. However those two approaches are in their current design not directly applicable to IoT networks. They need a lot of resources that smart objects, motes do not dispose of. Thus leading to the subject of our thesis to design a tool to permit network administrators to monitor their IoT networks.\\

As we will explain later on in more detail in Chapter \ref{ch:state_art}, the Internet of Things is the interconnection of smart objects. However, those smart objects differ in a few ways from the components of "traditional" networks, that are mainly composed of powerful computers and routers.\\

Those traditional networks usually have infinite power supply since they are directly plugged to power plugs. Smart objects may be embedded to another machine, in that case they may be supplied by power. However, some objects whose purpose is to sense humidity or temperature, for example in forests attached on trees, have limited battery life.  Such network are called \textit{Wireless sensor networks} (WSN) which are capable in an automatic fashion to report periodically the environmental conditions. Therefore, the objects cannot be in a running state perpetually. Occasionally, they will be left in a standby mode, when no information is gathered, meaning the object will not use its sensor to sense the physical world, or use its actuator to control it. That way the battery life is extended, and the object is in a running period only when data must be collected or must communicate with its neighbors. \\

Another important difference is processing capacity. Components of traditional networks are usually equipped with powerful processors and thus have a stronger processing capacity. On the other hand, smart objects first have a limited size, usually a few cubic centimeters. Consequently, those smart objects have small microprocessors and also have limited storage capacity. Having these limited capabilities greatly affects how the protocols are built, as an object receiving too many packets and having too much data to process will rapidly be overworked because of its relatively weak processing capacity. \\

As a result of all those main differences, IoT Networks impose different constraints than traditional networks, and lightweight protocols must be built, developed according to those constraints.\\

The software we wanted to develop was a Web page that would display the current topology of our Network we are administrating, with various information on the traffic flowing through the network, with information about senders and receivers of the packets, along with the volume of data exchanged through a certain period of time. Additional informations as current battery of motes or their topological links would be also relevant for our case. For that, we used \textit{Contiki} with \textit{Cooja} to program motes and simulate networks. We then used \textit{Nfdump}/\textit{Nfsen} as to provide a plug-in for visualizing the informations that we sent. Additional details on these technologies will be explained on latter chapters.\\

Chapter \ref{ch:state_art} will discuss in more details what the Internet of Things is and what lies under those technologies. Indeed, The IoT devices use their own operating systems, such as \textit{ContikiOS} and \textit{TinyOS} as well as their own protocols, to which we will take a look in that section. It is also important to know that the concept of interconnecting smart objects together has existed for more than a decade, the current state of the field is falling behind the state of traditional networks. Thus, we will see what kind of monitoring softwares exist for traditional networks. We will also take a look at existing tools for the IoT, see what they can achieve and how they suit the IoT.\\

The Chapter \ref{ch:solution} will be a presentation of the monitoring software we have worked on. As the second section depicts what kind of information our monitoring software must analyze, we will describe our approach to the solution we have come up with, the requirements for such a monitoring tool. Many technologies were available for use, thus we will present the tools that we have decided to use to build our software and the reasons behind the choices we made. We will more generally present the final software that we have developed throughout this year, more specifically how it monitors the data we are interested in analysing, and its particularities. \\

Since the main purpose of the thesis is monitoring, it implies that some analytics must be done to retrieve and analyse information and to draw some conclusion based on the observations. Chapter \ref{ch:analysis} will be a general analysis based on the results that our software has recorded. We will describe the tests we have performed on different IoT Networks and how they are implemented. We will also depict the performance of various IoT Networks, in terms of power usage and increase of traffic for instance.\\

Finally, we will discuss in Chapter \ref{ch:discussion} other solutions that could be brought to the table, and how our software could potentially grow if further work was done in the future. We will also discuss other aspects of networks such as security.
