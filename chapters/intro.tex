\chapter*{Introduction}
\addcontentsline{toc}{chapter}{Introduction}

Computers, the Internet. Those technologies are now everyday tools that a vast majority of people own and use these days, every day. They have shaped our habits, being able to communicate with everybody and anybody, through technologies other than traditional phones, with Facebook or WhatsApp for instance. The internet is now available on smaller objects, more famously on smartphones and smart tablets. But what if other electronic devices could be connected to the Internet and could communicate and exchange information? What if instead of having workers collect data in a particular location, electronic devices could do this task for us?\\

Such technologies go by the name of \textit{The Internet of Things}. The Internet of Things or IoT is defined as the interconnection of smart devices, or \textit{objects}. Typically, the IoT is composed of smart devices, also called motes which are able to communicate between themselves to exchange data for example. They can also be able to interact with the environment by sensing it with sensors (the action of observing the temperature of a house) or modify it by means of actuators (the action of turning the heat on of a house).\\

As our journey as students is coming to an end, the Master thesis presents itself as the most important step towards completion of our Master Degree.  Having chosen the Computer Network option in our Master Degree program, we were eager to make some deeper researches on networks and especially the IoT, fueled by the fact that it is a field that is not completely developed to its full potential. Hence, we have chosen the subject \textit{Network Administration Tool For the Internet of Things}, a subject suggested by Professor Sadre, with whom we have decided to work with on the thesis. \\

This thesis will explore the Internet of Things, specifically the monitoring of internetworks of such \textit{smart objects}.  In this regard, we must create a software that is able to analyse the traffic that flows through an IoT Network. The traffic we are interested in is composed of all the data, packets, messages exchanged between the objects or that comes out of the IoT network or have come into it. By visualize we mean showing with the help of graphical content such as curves and topologies, various information types that have been retrieved from the network. The information that we want to retrieve and process is traffic volume, battery consumption and level, and the general routing topology of the IoT network we are administrating.

\section*{Context}

One may ask what motivates such a research and development. IoT becomes more prevalent as of today but no tools exist to keep track of all the traffic that emerged from it. Some softwares do exist to monitor IoT but they focus mainly on data (sensors values) that comes out of it and overlook the internal traffic generated by such networks. But the more complex IoT networks will become, the more organizations will be in need of means to observe the network entirely as to, for example, improve its efficiency. Monitoring techniques already exist for traditional networks such as \textit{Netflow} and \textit{sFlow}. However, those two approaches are in their current design not directly applicable to IoT networks. They need a lot of resources that smart objects, motes, do not dispose of. Thus leading to the subject of our thesis to design a tool to permit network administrators to monitor their IoT networks.\\

As we will explain later on in more detail in Part \ref{part:state_art}, the Internet of Things is the interconnection of smart objects. However, those smart objects differ in a few ways from the components of "traditional" networks, that are mainly composed of powerful computers and routers.\\

Those traditional networks usually have infinite power supply since they are directly plugged to power plugs. Smart objects may be embedded to another machine, in that case they may be supplied by power. However, some objects whose purpose is to sense humidity or temperature, for example in forests attached on trees, have limited battery life.  Such network are called \textit{Wireless sensor networks} (WSN) which are capable in an automatic fashion to periodically report the environmental conditions. Therefore, the objects cannot perpetually be in a running state. Occasionally, they will be left in a standby mode, when no information is gathered and energy is harvested to fill up the battery, meaning the object will not use its sensor to sense the physical world, or use its actuator to control it. That way the battery life is extended, and the object is in a running period only when data must be collected or must communicate with its neighbors. \\

Another important difference is processing capacity. Components of traditional networks are usually equipped with powerful processors and thus have a stronger processing capacity. On the other hand, smart objects first have a limited size, usually a few cubic centimeters. Consequently, those smart objects have small microprocessors and also have limited storage capacity. Having these limited capabilities greatly affects how the protocols are built, as an object receiving too many packets and having too much data to process will rapidly be overworked because of its relatively weak processing capacity. \\

As a result of all those main differences, IoT Networks impose different constraints than traditional networks, and lightweight protocols must be built, developed according to those constraints.

\section*{Goal}

The goal of this thesis is to develop a monitoring tool for administrators to analyze and monitor their IoT networks.\\

To do so, the solution should be able to extract \textit{traffic informations} of an IoT network. This information could be the number of octets or packets from one sender to another receiver during a period of time. So as to be easy to use, this information should be shown via a \textit{GUI} (graphical user interface) by means of charts and graphs. \\

Also, because we are talking about Internet of Things, and in our case Wireless Sensors Networks, we should display the \textit{current topology of the network} being administrated. This would give information about how the routing is done. Another great asset would be able to show the sensors values and sensors status like their remaining battery.\\

Our solution should \textit{preserve the battery} and consume low energy. This is an important requirement as IoT devices and particularly WSNs rely on batteries. Some devices can work for a year without having their batteries replaced. Moreover, for certain IoT networks, the cost of replacing batteries is high (devices located on glacier, often not easily reached).\\

Another requirement would be that our solution should be \textit{easily integrated to existing infrastructure}. Organizations in need of monitoring their IoT networks should not need to review entirely their current solution for monitoring. Using existent protocols would thus be an asset. \\

A final requirement is that our software should be \textit{easily modifiable} so that users could tune it to suit their needs. All users do not share the same requirements. Our product should be a base product with essential features. But a user should have the possibility to freely add features.

\section*{Solution}

Our solution is composed of two parts. The first part is dedicated to the retrieval of information. The second part is the representation of the retrieved informations. \\

The retrieval of information has been done using the \textit{TinyIPFIX} protocol \cite{schmitt2016tinyipfix}, a light version of IPFIX (or Netflow v10) for Internet of Things. If you are familiar with how IPFIX works, TinyIPFIX is basically the same. We have exporters, in our case the IoT nodes, that collect packets and form flows (stream of packets sharing common attributes). The exporters send those flows in IPFIX records format to collectors that, like their name implies, collects the records and store them. An application can then have access to those records for further analyzing. The idea behind TinyIPFIX is for nodes to export the records in a lighter format than the original IPFIX format. This is done by compressing the IPFIX message header. The gateway, responsible to link the IoT network and the Internet will then reformat the TinyIPFIX message into an original IPFIX message. In doing so, organizations using IPFIX should not have to restructure their monitoring infrastructure. \\

The representation of information was done using \textit{Node.js} \cite{website:nodejs}, a server-side solution for JavaScript. We developed a Node.js solution responsible for collecting the IPFIX records and storing them. It was also used to display via web pages some topologies, graphs and charts that match the information retrieved. Node.js proved to be useful by its rapidity to develop with it. Additionally, the fact that it uses JavaScript, HTML and CSS makes it easier for other people to dive into our code and add features or modules. \\

The thesis is divided in four parts: State of the art, Solution, Analysis and Discussion.\\

Part \ref{part:state_art} will discuss in more details what the Internet of Things is and what lies under those technologies. Indeed, The IoT devices use their own operating systems, such as \textit{ContikiOS} and \textit{TinyOS} as well as their own protocols, to which we will take a look in that section. It is also important to know that the concept of interconnecting smart objects together has existed for more than a decade, the current state of the field is falling behind the state of traditional networks. Thus, we will see what kind of monitoring softwares exist for traditional networks. We will also take a look at existing tools for the IoT, see what they can achieve and how they suit the IoT.\\

Part \ref{part:solution} will be a presentation of the monitoring software we have worked on. As the second section depicts what kind of information our monitoring software must analyze, we will describe our approach to the solution we have come up with, the requirements for such a monitoring tool. Many technologies were available for use, thus we will present the tools that we have decided to use to build our software and the reasons behind the choices we made. We will more generally present the final software that we have developed throughout this year, more specifically how it monitors the data we are interested in analyzing, and its particularities. \\

Since the main purpose of the thesis is monitoring, it implies that some analytics must be done to retrieve and analyse information and to draw some conclusion based on the observations. Part \ref{part:analysis} will be a general analysis based on the results that our software has recorded. We will describe the tests we have performed on different IoT Networks and how they are implemented. We will also depict the performance of various IoT Networks, in terms of power usage and increase of traffic for instance.\\

Finally, we will discuss in Part \ref{part:discussion} other solutions that could be brought to the table, and how our software could potentially grow if further work was done in the future. We will also discuss other aspects of networks such as security.
