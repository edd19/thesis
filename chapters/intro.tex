\chapter*{Introduction}
\addcontentsline{toc}{chapter}{Introduction}

\begin{itemize}
	\item Subject of master thesis
	\item Why this master thesis
	\item Talk about different sections of the thesis
\end{itemize}

Computers, the Internet. Those technologies are now everyday tools that a vast majority of people own and use these days, every day. They have shaped our habits, being able to communicate with everybody and anybody, through technologies other than traditional phones, with Facebook or WhatsApp for instance. The internet is now available on smaller objects, more famously on smartphones, and smart tablets. But what if other electronic devices could be connected to the Internet, and could communicate and exchange information? (What if instead of having workers collect data in a particular location, electronic devices could do this task for us?)\\

\todo{+ Développer?}
Such technologies go by the name of \textit{The Internet of Things}. The Internet of Things is defined as the interconnection of smart devices, or \textit{objects}. \\

As our journey as students is coming to an end, the Master thesis presents the most important step towards completion of our Master Degree.  Having chosen the Computer Network option in our Master Degree program, we were eager to make some deeper researches on networks and especially the IoT, fueled by the fact that it is a field that is not completely developed to its full potential. Hence, we have chosen the subject \textit{Network Visualization for the IoT}, a subject suggested by Professor Sadre, with whom we have decided to work with on the thesis. \\

\todo{FAUT-IL DIRE "FOURTH SECTION" OU "THE OBJECTIVE - ANALYSIS - DISCUSSION" SECTION?}
This thesis will explore the Internet of Thing, specifically the monitoring of internetworks of such \textit{smart objects}. The first chapter (Objective section?) will depict the purpose and the goal of the thesis. We will discuss the type of traffic we are interested in while monitoring the IoT Networks. We will also talk about the constraints of studying the IoT world, specifically the fact that the compononents of those networks are not plugged computers but small devices on low-battery capacity and low-powered processors. (As the Internet of Things is still not implanted to the everyday world and will probably be under heavier development in the years to come, there is not a large number of softwares already developed.).\\

The second chapter will discuss in more details what the Internet of Things is and what lies under those technologies. Indeed, The IoT devices use their own operating systems, such as \textit{ContikiOS} and \textit{TinyOS} as well as their own protocols, to which we will take a look in that section. It is also important to know that the concept of interconnecting smart objects together has existed for more than a decade, the current state of the field is falling behind the state of traditional networks. Thus, we will see what kind of monitoring softwares exist for traditional networks. We will also take a look at existing tools for the IoT, see what they can achieve and how they suit the IoT.\\

The third chapter will be a presentation of the monitoring software we have worked on. As the second section depicts what kind of information our monitoring software must analyze, we will describe our approach to the solution we have come up with, the requirements for such a monitoring tool. Many technologies were available for use, thus we will present the tools that we have decided to use to build our software and the reasons behind the choices we made. We will more generally present the final software that we have developed throughout this year, more specifically how it monitors the data we are interested in analysing, and its particularities. \\

Since the main purpose of the thesis is monitoring, it implies that some analytics must be done to retrieve and analyse information and to draw some conclusion based on the observations. The fourth section will be a general analysis based on the results that our software has recorded. We will describe the tests we have performed on different IoT Networks and how they are implemented. We will also depict the performance of various IoT Networks, in terms of power usage and increase of traffic for instance.\\

Finally, we will discuss in the fifth section other solutions that could be brought to the table, and how our software could potentially grow if further work was done in the future. We will also discuss other aspects of networks such as security.
